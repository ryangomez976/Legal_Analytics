\section{Discussion}

All of the models suffer from severe over-fitting, and efforts to reduce over fitting ultimately had little effect. I attribute most of the over-fitting to the sheer amount of noise in the data. The CBAs were completely unstructured, with no obvious way to parse the text, as each district provides its own unique formatting for each document. On average, a CBA contains approximately 23,489.5 words,\footnote{\hspace{2ex}This was calculated via Terminal, after setting the working directory, with the command: \texttt{find . -type f -print0 | xargs -0 cat | wc -w}, which returned 6,342,165 words for 270 documents.} and the \textt{rpart} classifier is attempting to locate very specific clauses from each CBA. One would expect that the CBAs are, as a whole, very similar and address similar labor issues, despite also having unique clauses; a cursory look at random CBAs confirms that this is true. For this model, it means that the \texttt{rpart} algorithm is likely picking up on irrelevant variables, instead of the specific clause that I am looking for. 

This finding is largely confirmed by a proximity matrix that was created using the \texttt{bigrf} package.\footnote{\hspace{2ex}\textit{See} Figure 2 for proximity matrix.} This package uses random forests, or many decision trees as opposed to one, in order to classify the CBAs as having or not having a specified clause. A proximity matrix, which, broadly speaking, measures the similarity or dissimilarity between documents was then plotted. In the proximity matrix for \texttt{sideletterpresence}, we can visually see that the decision trees are finding irrelevant similarities and dissimilarities between documents that have side letter agreements and those that don’t. This is likely due to the large amount of noise in each document. 

Despite the problems with the data and model, my hypothesis---that by predicting which clauses a school district has in its CBA, we can characterize that school district---and the practical and academic implications that stem from that has been, at least partially, anecdotally, confirmed. Several superintendents that I spoke with during this study were able to confirm that they in fact have clauses in the district’s CBA that are not applicable to the district as it currently is. Moreover, superintendents also seem to agree that certain clauses, such as “participative decision making,” are essential for student success and workplace satisfaction, and further the government-backed, “teach to lead,” initiative.\footnote{\hspace{2ex}\textit{See} \textsc{U.S. Dept. of Education}, \textit{Teach to Lead: Advancing Teacher Leadership} (March 14, 2014), \url{https://www.ed.gov/news/speeches/teach-lead-advancing-teacher-leadership}.} 

Moving forward with this project will mostly involve formatting the CBAs in XML, which enables users to “tag” documents and then carry out tasks associated with those tags. Doing this should vastly reduce the amount of noise, since clauses could be specifically targeted, and similar clauses could then be identified and traced back to the document and district. Simply being able to target clauses, paragraphs, or even articles, would vastly improve this model’s performance.
