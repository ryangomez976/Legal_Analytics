\section{Introduction}
\subsection{The Data}

The relevant population for this project consists of all school districts in California that have collective bargaining agreements (“CBA”) with the teacher’s union of that school district. Not every school district in California has an associated teacher’s union and CBA; however, most do. The CBAs are public documents, but schools are not required to report the contracts to any private or government entity. This presents unique challenges in gathering the CBAs from each school district that has one. 
	
In order to overcome the challenges associated with collecting each district’s CBA, I elected to use data available through California’s Department of Education in order to obtain an email list of each superintendent in California. This list includes every school in California, both active and closed, as well as schools without websites or contact information, which might indicate that it is either an incredibly small district or other issues. Using the dplyr package,\footnote{\hspace{2ex}Appendix 1.} I first selected variables that might be relevant to my search for school district contacts. I then further refined the data to only include unique school districts, since the district as a whole, rather than individual schools, contract with the associated teacher’s union. I then filtered that data by active school districts, districts that have websites, and districts that have email contact information for the superintendent. 
	
I was left with 988 school districts with superintendent email contacts, out of the 1,028 school districts in California.\footnote{\hspace{2ex}\textsc{California Dept. of Education,} \url{http://www.cde.ca.gov/ds/sd/cb/ceffingertipfacts.asp}, (last visited May 6, 2015).}  I then emailed all 988 superintendents requesting the CBA between the district and its teachers, and received 281 CBAs in response (28.4\% response rate). The CBAs came in a variety of file formats, some of which were scanned PDFs. In order to use the CBAs, I had to run OCR over them with Abbyy FineReader in order to get the each CBA into a machine-readable, plain HTML format that could be processed using the R program.\footnote{\hspace{2ex}For information on optical character recognition, \textit{see} \url{http://en.wikipedia.org/wiki/Optical_character_recognition}.}  Abbyy could not process several CBAs as they were password protected, and I ended up with 270 CBAs in total, or 26.2\% of all school districts in California.
	
In order to further process and clean the data, I placed the CBAs into a text corpus,\footnote{\hspace{2ex}For more information on text corpus, \textit{see} \url{http://en.wikipedia.org/wiki/Text_corpus}.}  transformed the contents to lowercase, removed numbers, punctuation and whitespace, and further removed stop words\footnote{\hspace{2ex}For information on stop words, \textit{see} \url{http://en.wikipedia.org/wiki/Stop_words}.} that appear within the tm package as well as stop words associated with HTML.\footnote{\hspace{2ex}\textit{See}, Figure 1 for a word cloud from all 270 CBAs.}  After cleaning preparing the corpus, I placed it into a document-term matrix.\footnote{\hspace{2ex}For more information on document-term matrices, \textit{see} \url{http://en.wikipedia.org/wiki/Document-term_matrix}.}

\subsection{Research Problem}

For this project, I wanted to explore the possibility of identifying clauses contained within a CBA that might indicate a unique attribute of the school district. For instance, certain school districts have “participative decision making” clauses that give teachers a voice in the development and improvement of instructional programs. School districts that have this clause in the CBA likely give more classroom autonomy to teachers, which might have discernible effects on each the school’s Academic Performance Index (“API”) scores.\footnote{\hspace{2ex}\textsc{California Dept. of Education}, \url{http://www.cde.ca.gov/Ta/ac/ap/} (last visited May 6, 2015).}

Being able to locate clauses that indicate unique attributes of a school district has both practical and academic implications. Academically, a school can be profiled based on certain clauses, and further research could be conducted to determine correlations with API scores, teacher autonomy, workplace satisfaction, or other important measures of school success. Practically, certain clauses might suggest an attribute of the school district that is no longer true, which in turn might suggest that the clause is superfluous. This could cut down attorney time by automatically identifying superfluous clauses, or by reducing negotiating time as the parties can focus on clauses that are actually applicable. 

In order to explore this idea, I used decision trees\footnote{\hspace{2ex}For information on how decision trees can be used for machine learning, \textit{see} \url{http://en.wikipedia.org/wiki/Decision_tree_learning}.} through the package \texttt{rpart} in order to predict whether a CBA contained a certain clause or not. The three clauses that I sought to predict were: presence or absence of a side letter agreement, presence or absence of a professional workday, and presence or absence of a defined workday. 
